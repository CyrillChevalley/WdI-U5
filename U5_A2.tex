\documentclass[11pt,a4paper]{article}
\usepackage[ngerman]{babel}
\usepackage[utf8]{inputenc}
\usepackage{amsmath}

\title{CS 102 \LaTeX Übung}
\author{Cyrill Chevalley}
\date{\today}

\begin{document}
\maketitle

\section{Das ist der erste Abschnitt}
Hier könnte auch anderer Text stehen.

\section{Tabelle}
Unsere wichtigsten Daten finden Sie in Tabelle 1.
	\begin{table}[h]
	\begin{centering}
	\begin{tabular}{c|c|c|c}
	&Punkte erhalten&Punkte möglich& \% \\
	\hline
	Aufgabe 1&2&4&0.5 \\
	Aufgabe 2&3&3&1 \\
	Aufgabe 3&3&3&1 \\
	\end{tabular}
	\caption{Diese Tabelle kann auch andere Werte beinhalten.}
	\end{centering}
	\end{table}
	
\section{Formeln}
\subsection{Pythagoras}
Der Satz des Pythagoras errechnet sich wie folgt: $a^{2} + b^2 = c^2$.
Daraus können wir die Länge der Hypothenuse wie folgt berechnen: 
$c = \sqrt{a^2 + b^2} $

\subsection{Summen}	
Wir können auch die Formel für die Summe angeben:
\begin{equation}
s=\sum \limits_{i=1}^{N} i = \frac{n \ast (n+1)}{2}
\end{equation}

CPAeby was here am 2014-11-13

\end{document}
